\section{conclusion}
\par In this paper, a novel approach of adaptively the spark parameter configurations for big data analytics is proposed, and a model based on binary classification and multi classification is established. Furthermore, several common machine learning algorithms for configuration parameter tuning are explored based on the model. Experimental results show that CPSO has good accuracy and computational performance across diverse workloads for binary classification and multi-classification. And experimental results also show that our approach is effective and available to tune the configuration parameters of Spark platform. An average of 36
\% performance improvement can be got with the proposed method. Moreover, with the increase of input data, the effect of performance improvement is more obvious.
\par At the same time, the proposed approach is more robust and flexible as it is easier to adapt to changes. Experimental results demonstrate that it is possible to build an effective performance auto-tuner for Spark in a black box manner, that is, in manner of only using observations from the system and without getting its internals.
\par In the future, we will conduct further research on parameter selection in order to select more suitable parameters. And, we will also consider more parameters affecting on the performance,  try to cover all of spark parameters.

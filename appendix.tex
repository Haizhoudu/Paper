\section{Appendix}\label{sec:appendix}

\begin{table*}[!htbp]
\caption{Selected Parameters In the Hummingbird} \footnotesize \label{tab:parameters} 
\begin{center}
    \begin{tabular}{ | c | c  | p{4cm} | c | p{5cm} |}
    \hline
    Parameter Name & Default & Meaning & Rules of Thumb & Importance\\ 
    \hline
    spark.driver.memory & 512m & Amount of memory to use for the driver process. & 1g-4g & Directly affect computing performace of spark memory.\\ 
    \hline
    spark.driver.maxResultSize & 1g & Each Spark action in the total size of the serialization results for all partitions. & 1g-8g & It represents a fixed memory overhead per reduce task.\\ 
    \hline
    spark.executor.memory &  1g &Amount of memory to use per executor process.   & 2g-8g & Directly affect computing performace of spark memory.\\
    \hline
     spark.driver.cores & 1 & Number of cores to use for the driver process. & 1-8. & Directly affect the parallelism of the program.\\
    \hline
      spark.executor.cores &  1 & Number of cores to use  per executor process. & 10-40 & Directly affect the parallelism of the program.\\
    \hline
     spark.shuffle.file.buffer &  32k & Size of the in-memory buffer for each shuffle file output stream. & 10-40 & These buffers reduce the number of disk seeks and system calls made in creating intermediate shuffle files.\\
    \hline
     spark.reducer.maxSizeInFlight &  48m & Maximum size of map outputs to fetch simultaneously from each reduce task.  & 24k-96k & This represents a fixed memory overhead per reduce task. \\
    \hline
     spark.shuffle.compress &  true & Whether to compress map output files. & true/false & It will speed up the shuffle action.\\
    \hline
     spark.shuffle.spill.compress &  true & Whether to compress data spilled during shuffles. & true/false & It will speed up the shuffle action.\\
    \hline
     spark.broadcast.compress &  true & Whether to compress broadcast variables before sending them. & true/false & It will speed up the broadcast action.\\
    \hline
     spark.io.compression.codec &  lz4 & The codec used to compress internal data such as RDD partitions, event log, broadcast variables and shuffle outputs. & lz4/lzf/snappy. &  It will speed up all of I/O action.\\
    \hline
     spark.rdd.compress &  false & Whether to compress serialized RDD partitions. & true/fasle & It can save substantial space at the cost of some extra CPU time.\\
    \hline
     spark.default.parallelism &  0.6 & Default number of partitions in RDDs returned by transformations. & 0.4-0.8 & It will add RDD parallelism when user operate join, reduceByKey, and parallelize.\\
    \hline
     spark.memory.fraction &  0.6 & Fraction of (heap space - 300MB) used for execution and storage & 0.4-0.8 & The purpose of this config is to set aside memory for internal metadata, user data structures, and imprecise size estimation in the case of sparse, unusually large records. \\
    \hline
     spark.memory.storageFraction &  0.5  & Amount of storage memory immune to eviction, expressed as a fraction of the size of the region set aside by it. & 0.3-0.8 & The higher this is, the less working memory may be available to execution and tasks may spill to disk more often.\\
       \hline
     spark.broadcast.blockSize &  4m & Size of each piece of a block for TorrentBroadcastFactory. & 2m-8m & It will affect parallelism during broadcast.\\
      \hline
    spark.files.useFetchCache &  true & File fetching will use a local cache that is shared by executors that belong to the same application, which can improve task launching performance when running many executors on the same host. & true/false & This optimization may be disabled in order to use Spark local directories that reside on NFS filesystems.\\    
  \hline
     spark.files.openCostInBytes &  4M & The estimated cost to open a file, measured by the number of bytes could be scanned in the same time. & true/false &the partitions with small files will be faster than partitions with bigger files.\\
  \hline
     spark.storage.memoryMapThreshold &  2m & Size of a block above which Spark memory maps when reading a block from disk. & 1m-4m & It prevents Spark from memory mapping very small blocks.\\
  \hline
    spark.rpc.message.maxSize & true & Maximum message size (in MB) to allow in "control plane" communication. & true/false & It will speed up the network action when  you are running jobs with many thousands of map and reduce tasks.\\
  \hline
    spark.speculation.quantile &  0.75 & Fraction of tasks which must be complete before speculation is enabled for a particular stage. & 0.45-0.9 & It will speed up the task.\\
  \hline
     spark.dynamicAllocation.enabled &  true & Whether to use dynamic resource allocation, which scales the number of executors registered with this application up and down based on the workload. & true/false & It will increase the resource when we are handling a workload .\\
   \hline    
    \end{tabular}
\end{center}	
\end{table*}




